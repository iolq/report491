\chapter{\ifproject%
      \ifenglish Experimentation and Results\else การทดลองและผลลัพธ์\fi
  \else%
      \ifenglish System Evaluation\else การประเมินระบบ\fi
  \fi}



% \section{\ifenglish put the name here \else การสำรวจความคิดเห็นต่อเกมสยองขวัญ\fi}
% จากผลสำรวจกลุ่มเป้าหมายทั้งหมด 47 คน ได้ผลสรุปดังนี้
% \begin{itemize}
%     \item สาเหตุที่ผู้เล่นนั้น ไม่สามารถเล่นเกมจนจบได้นั้น คือ อันดับ 1 ไม่มีเวลาเล่น คิดเป็น 74.5$\%$, อันดับ 2 ไม่มีคนเล่นด้วย คิดเป็น 53.2$\%$, และอันดับ 3 คู่ร่วม คือ เนื้อเรื่องไม่มีความน่าสนใจ และ มีเกมอื่นที่น่าสนใจมากกว่า คิดเป็น 18.3$\%$
%     \item เกมแนวสยองขวัญควรมีอะไรบ้าง คือ อันดับ 1 มีฉากนองเลือด คิดเป็น 85.1$\%$, อันดับ 2 มีบรรยากาศที่ดูอึดอัด คิดเป็น 76.6$\%$, และอันดับ 3 ความมืด คิดเป็น 74.5$\%$
%     \item สิ่งที่ไม่ถูกใจในเกมสยองขวัญคือ มีความยากในการเล่นสูงเกินไป มีความซ้ำซากจำเจ ต่อสู้กับผีไม่ได้ และมีฉาก jump scare มากเกินไป
% \end{itemize}
% \begin{figure}[h]
%     \centering
%     \includegraphics[width=\textwidth]{Images/Screenshot 2023-10-06 192027.png}
%     \caption{ภาพผลสำรวจความคิดเห็นที่มีต่อเกมสยองขวัญ}\label{Empatize}
% \end{figure}
% \section{\ifenglish put the name here \else การวางแผนการประเมินผลสัมฤทธิ์ของโปรเจค\fi}
% \begin{enumerate}
%     \item นำ prototype มาทดสอบกับกลุ่มผู้ใช้งาน โดยให้เวลาในการทดสอบของแต่ละคน 15 นาที โดยใช้อุปกรณ์ที่เราได้จัดเตรียมไว้
%     \item สอบถามปัญหา จากการใช้งาน prototype
%     \item นำแบบทดสอบให้ทำ หลังการทดสอบ โดยมีเนื้อหาใจความดังนี้
%     \begin{itemize}
%         \item ท่านให้ความน่ากลัวของเกมนี้ อยู่ในระดับไหน 0-10 โดยที่ 0 คือ ไม่เห็นน่ากลัวเลย, 3 คือ น่ากลัวนิดๆหน่อยๆ, 5 คือ ก็น่ากลัวนะ, 7 คือ น่ากลัวมากๆ, และ 10 คือ จะน่ากลัวไปไหน
%         \item ท่านให้ความสนุกอยู่ระดับ 0 - 10 โดยที่ 0 คือ ไม่สนุกเลย, 3 คือ สนุกนิดหน่อย, 5 คือ สนุกปานกลางๆ, 7 คือ สนุกมาก, และ 10 คือ สนุกที่สุด
%         \item มีเนื้อเรื่องส่วนใด ที่ท่านชอบมากที่สุด อธิบาย
%     \end{itemize}
% \end{enumerate}

\section{\ifenglish put the name here\else วัตถุประสงค์ของการทดสอบ\fi}
ทดสอบว่าผู้เล่นนั้นได้รับความสนุกในด้านความน่ากลัว

\section{ขั้นตอนและวิธีการทดสอบ}
ผู้พัฒนาได้ทำการทดสอบกับผู้เล่นจำนวน 10-11 คน โดยเป็นนักศึกษาคณะวิศวกรรมศาสตร์ สาขาวิศวกรรมคอมพิวเตอร์ โดยแบ่งการทดสอบเป็น 2 กรณี
\begin{enumerate}
    \item กรณีแบบออนไลน์
    \begin{enumerate}
        \item ผู้เล่นทำการเปิดโปรแกรม Discord จากนั้นทำการแชร์หน้าจอและทำการเล่นเกมจนสิ้นสุด
        \item ผู้พัฒนาทำการเผ้าดูพฤติกรรมของผู้เล่นที่มีต่อตัวเกม
        \item หลังจากผู้เล่นทำการเล่นเกมจนสิ้นสุด ผู้เล่นทำแบบทดสอบที่เราได้ตระเตรียมไว้
    \end{enumerate}
    \item กรณีแบบออนไซต์
    \begin{enumerate}
        \item ผู้เล่นทำการเล่นเกมจนสิ้นสุด
        \item ผู้พัฒนาทำการเผ้าดูพฤติกรรมของผู้เล่นที่มีต่อตัวเกม
        \item หลังจากผู้เล่นทำการเล่นเกมจนสิ้นสุด ผู้เล่นทำแบบทดสอบที่เราได้ตระเตรียมไว้
    \end{enumerate}
\end{enumerate}

\section{\ifenglish put the name here \else ผลการทดสอบผลสัมฤทธิ์ของผู้เข้าร่วม\fi}
จากการทดสอบกับผู้เล่นจำนวน 10-11 คน โดยมีผลการทดสอบเป็นดังนี้

\begin{enumerate}
    \item หลังจากผู้เล่นทำการเล่นเกมจนเสร็จสิ้นนั้น คิดว่าเกมมีความยากง่ายเพียงใด 50$\%$ ของผู้เล่นบอกว่า เกมนั้นมีความยากระดับปานกลาง ไม่ยากและก็ไม่ง่าย 30$\%$ ของผู้เล่นบอกว่า เกมนั้นมีความยากต่อการเล่น และ 20$\%$ ของผู้เล่นบอกว่า เกมนั้นมีความง่าย
    \item ระบบ sanity ส่งผลต่ออุปสรรคในการเล่นเพียงใด 45.5$\%$ ของผู้เล่นบอกว่า ระบบ Sanity นั้น เป็นอุปสรรคต่อผู้เล่นปานกลาง 36.4$\%$ ของผู้เล่นบอกว่า ระบบ Sanity นั้น เป็นอุปสรรคต่อผู้เล่นมาก และ 18.2$\%$ ของผู้เล่นบอกว่า ระบบ Sanity นั้น เป็นอุปสรรคต่อผู้เล่นน้อย
    \item ความรู้สึกหลังเล่น ความน่ากลัวของเกม 54.5$\%$ ของผู้เล่นบอกว่า เกมมีความน่ากลัวน้อย 36.4$\%$ ของผู้เล่นบอกว่า เกมมีความน่ากลัวปานกลาง และ 9.1$\%$ ของผู้เล่นบอกว่า เกมมีความน่ากลัวมากที่สุด
    \item ความรู้สึกหลังเล่น ความกดดันของเกม 45.5$\%$ ของผู้เล่นบอกว่า เกมมีความกดดันปานกลาง 18.2$\%$ ของผู้เล่นบอกว่า เกมมีความกดดันมาก 18.2$\%$ ของผู้เล่นบอกว่า เกมมีความกดดันน้อย และ 18.2$\%$ ของผู้เล่นบอกว่า เกมมีความกดดันน้อยมาก
    \item เสียงประกอบของเกม 63.6$\%$ ของผู้เล่นบอกว่า เสียงประกอบเข้ากับสถานการณ์มาก 27.3$\%$ ของผู้เล่นบอกว่า เสียงประกอบเข้ากับสถานการณ์ปานกลาง และ 9.1$\%$ ของผู้เล่นบอกว่า เสียงประกอบเข้ากับสถานการณ์น้อยมาก
    \item ความเข้ากันของตัวละครหลักและฉาก 45.5$\%$ ของผู้เล่นบอกว่า ตัวละครเข้ากับฉากมาก 27.3$\%$ ของผู้เล่นบอกว่า ตัวละครเข้ากับฉากปานกลาง 18.2$\%$ ของผู้เล่นบอกว่า ตัวละครเข้ากับฉากเล็กน้อย และ 9.1$\%$ ของผู้เล่นบอกว่า ตัวละครเข้ากับฉากมากที่สุด
    \item ความเข้ากันของตัวละครแพรวและฉาก 45.5$\%$ ของผู้เล่นบอกว่า ตัวละครเข้ากับฉากปานกลาง 36.4$\%$ ของผู้เล่นบอกว่า ตัวละครเข้ากับฉากมาก 9.1$\%$ ของผู้เล่นบอกว่า ตัวละครเข้ากับฉากมากที่สุด และ 9.1$\%$ ของผู้เล่นบอกว่า ตัวละครเข้ากับฉากเล็กน้อย
    \item ความเข้ากันของตัวละครศัตรูและฉาก 36.4$\%$ ของผู้เล่นบอกว่า ตัวละครเข้ากับฉากมากที่สุด 27.3$\%$ ของผู้เล่นบอกว่า ตัวละครเข้ากับฉากมาก 27.3$\%$ ของผู้เล่นบอกว่า ตัวละครเข้ากับฉากปานกลาง และ 9.1$\%$ ของผู้เล่นบอกว่า ตัวละครเข้ากับฉากน้อยมาก
    \item 
\end{enumerate}

\section{\ifenglish put the name here\else สรุปผลการทดสอบผลสัมฤทธิ์\fi}
จากการประเมิน
