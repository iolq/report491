\maketitle
\makesignature

\ifproject
    \begin{abstractTH}
        % เขียนบทคัดย่อของโครงงานที่นี่

        เกมแนวสยองขวัญ เป็นเกมที่จะมีเรื่องราวให้ผู้เล่นได้สวมบทบาทและเผชิญหน้ากับสิ่งที่น่าหวาดกลัว ความกดดัน และเอาตัวรอดจากสถานการณ์ต่าง ๆ ในเกม 
        ซึ่งเป็นแนวเกมที่ได้รับความนิยมในยุคปัจจุบัน โดยในโครงงานนี้ จะเป็นการพัฒนาเกมแนวสยองขวัญ ที่มีมุมมองบุคคลที่ 3 แบบด้านข้าง โดยผู้เล่นจะได้รับบทบาทเป็นนักเรียนคนหนึ่ง 
        ที่มีเพื่อนสนิทพยายามฆ่าตัวตาย แต่ไม่สำเร็จ และนอนไม่ได้สติอยู่ในโรงพยาบาล ผู้เล่นจะต้องค้นหาความจริง โดยการไขปริศนาต่าง ๆ และต้องเอาตัวรอดจากอุปสรรคที่คอยรบกวนผู้เล่น 
        ไม่ให้ไปถึงความจริง และต้องการจะปลิดชีพผู้เล่น ระบบของเกมจะถูกออกแบบให้ผู้เล่นจะต้องคอยบริหารค่าพลังงานที่ใช้ในการทำกิจกรรมต่างๆ ค่าสติที่มีผลกับการรับรู้ของตัวละคร 
        และเวลาที่มีอยู่อย่างจำกัด ซึ่งข้อจำกัดเหล่านี้จะส่งผลให้ผู้เล่นมีความกดดันในการตัดสินใจเลือกการกระทำ สำหรับการดำเนินเนื้อเรื่องและฉากจบ จะมีการเปลี่ยนแปลงไปตามการเลือก ข้อมูล 
        และไอเทมที่ได้รับในระหว่างเล่นเกม โดยเกมนี้ถูกพัฒนาขึ้นด้วยโปรแกรม Unity และภาษา C$\#$ เกมถูกออกแบบมาสำหรับผู้ที่มีอายุ 18 ปีขึ้นไป และเล่นบน PC โดยผู้พัฒนาคาดหวังว่า 
        ผู้เล่นจะได้รับความบันเทิง ความสนุก และข้อคิดจากเรื่องราวที่ปรากฏภายในเกม
        % การเขียนรายงานเป็นส่วนหนึ่งของการทำโครงงานวิศวกรรมคอมพิวเตอร์
        % เพื่อทบทวนทฤษฎีที่เกี่ยวข้อง อธิบายขั้นตอนวิธีแก้ปัญหาเชิงวิศวกรรม และวิเคราะห์และสรุปผลการทดลองอุปกรณ์และระบบต่างๆ
        % \enskip อย่างไรก็ดี การสร้างรูปเล่มรายงานให้ถูกรูปแบบนั้นเป็นขั้นตอนที่ยุ่งยาก
        % แม้ว่าจะมีต้นแบบสำหรับใช้ในโปรแกรม Microsoft Word แล้วก็ตาม
        % แต่นักศึกษาส่วนใหญ่ยังคงค้นพบว่าการใช้งานมีความซับซ้อน และเกิดความผิดพลาดในการจัดรูปแบบ กำหนดเลขหัวข้อ และสร้างสารบัญอยู่
        % \enskip ภาควิชาวิศวกรรมคอมพิวเตอร์จึงได้จัดทำต้นแบบรูปเล่มรายงานโดยใช้ระบบจัดเตรียมเอกสาร
        % \LaTeX{} เพื่อช่วยให้นักศึกษาเขียนรายงานได้อย่างสะดวกและรวดเร็วมากยิ่งขึ้น
    \end{abstractTH}

    \begin{abstract}

        
        A horror-themed game is a game where players take on roles and confront terrifying situations, pressure, and strive various scenarios in the game.
        This genre pf games has gained popularity in recent time. In this project, the development will focus on creating a third-person horror game perspective.
        Plyers will take on the role of a student who, after surviving an attempted suicide by their close friend, losesconsciousness and wake up in a hospital.
        Plyers will have to uncover the truth by solving various mysteries and obstacles that hider thier path to the truth.
        They must also navigate through challenges that attempt to thwart thier progress and endanger thier lives, as someone seeks to eliminate them.
        The game system is designed so that players must manage their energy levels of various activities, their character's awareness, and the limited available time. 
        These constraints lead players to experience pressure when making decisions. 
        The progression of the storyline and the ending scenes will change based on the choices made, information, and items obtained during gameplay will also influence these outcomes.
        This game is developed using the Unity and C$\#$ language.
        The game is designed for players aged 18 and above. And it is playable on PC.
        The developers aim to provide enterment, fun, and throught-provoking experiences through the storyline presented within the game.

    \end{abstract}

    \iffalse
        \begin{dedication}
            This document is dedicated to all Chiang Mai University students.

            Dedication page is optional.
        \end{dedication}
    \fi % \iffalse

    \begin{acknowledgments}
        การพัฒนาเกม Attention please เพื่อเป็นโปรแกรมสร้างความบันเทิงให้กับผู้เล่น และได้รับคำปรึกษา

        ขอขอบคุณ ผศ.ดร.กานต์ ปทานุคม ที่ให้คำปรึกษา แนะนำ และให้การสนับสนุน ขอขอบคุณ รศ.ดร.ศักดิ์กษิต ระมิงค์วงศ์ 
        และ ผศ.ดร.นวดนย์ คุณเลิศกิจ ที่ให้คำแนะนำ
        \texttt{acknowledgment} environment.

        \acksign{2020}{5}{25}
    \end{acknowledgments}%
\fi % \ifproject

\contentspage

\ifproject
    \figurelistpage

    \tablelistpage
\fi % \ifproject

% \abbrlist % this page is optional

% \symlist % this page is optional

% \preface % this section is optional
