\chapter{\ifenglish Introduction\else บทนำ\fi}

\section{\ifenglish Project rationale\else ที่มาของโครงงาน\fi}
% ในปัจจุบัน เกมแนวสยองขวัญได้รับความนิยม โครงงานนี้จึงเป็นการพัฒนาเกมสยองขวัญ ที่มีมุมมองแบบบุคคลที่ 3 ด้านข้าง 
% และต้องการนำเสนอเรื่องราวที่ผู้พัฒนาร่วมกันนำเสนอในรูปแบบวิดีโอเกม


ในปัจจุบัน อุตสาหกรรมเกมเติบโตอย่างรวดเร็ว เนื่องจากเป็นอุตสาหกรรมที่ขึ้นอยู่กับพื้นฐานนวัตกรรมและเทคโนโลยีที่ตอบสนองต่อผู้คน ในปี 2019 รายได้ของอุตสาหกรรมเกมมีมูลค่าราว 1.46 แสนล้านดอลล่าร์สหรัฐ จากข้อมูลของบริษัทวิเคราะห์
อุตสาหกรรมเกม Newzoo ซึ่งมีมูลค่าสูงกว่ามูลค่าอุตสาหกรรมเพลงและ Box office ทั่วโลก 2 และ 3 เท่าตามลำดับ ในประเทศไทย มูลค่าอุตสาหกรรมเกมเติบโตอย่างต่อเนื่อง 


ในหมู่เกมที่ได้รับความนิยมและถูกผลิตออกมาอย่างมาก คือ เกมแนวสยองขวัญ ซึ่งมีเกมที่ได้รับความสนใจอย่างมากเช่น Outlast ของผู้พัฒนาอย่าง Red Barrel, Until Down ของผู้พัฒนา Supermassive Games, 
Fatal Frame ของผู้พัฒนา KOEI TECMO GAMES, และ The Coma 2: Vicious sister ของผู้พัฒนา Devespresso Games ที่ได้รับ รางวัล Indie Game Award Grand Prix และรางวัล Indie Game Award Best Narration 
ในปี 2020 ผู้พัฒนาจึงสนใจที่จะหยิบเกมแนวสยองขวัญนำมาพัฒนา โดยมีแนวสยองขวัญ ที่มีมุมมองแบบบุคคลที่ 3 แบบด้านข้าง และนำเสนอเรื่องราวที่ผู้พัฒนาร่วมกันออกแบบ

\section{\ifenglish Objectives\else วัตถุประสงค์ของโครงงาน\fi}

% ผู้เล่นได้รับรู้และตระหนักถึงเรื่องของความรุนแรง การกลั่นแกล้ง การเพิกเฉยต่อการกลั่นแกล้ง และผลที่ตามมาของการกระทำเหล่านั้น
\begin{itemize}
    \item ผู้เล่นได้รับความบันเทิงจากความสยองขวัญของเกม ไม่ว่าจะเป็นเรื่องฉาก บรรยากาศ เสียง
    \item พัฒนาเกมต้นแบบ เพื่อความบันเทิง
    \item ผู้เล่นได้รับรู้เนื้อเรื่องที่ผู้จัดทำแต่งขึ้น
\end{itemize}

\section{\ifenglish Project scope\else ขอบเขตของโครงงาน\fi}

\subsection{\ifenglish User scope\else ขอบเขตด้านผู้ใช้งาน\fi}
\begin{itemize}
    \item บุคคลที่มีอายุ 18 ปีขึ้นไป และมีความสนใจในการเล่นเกมสยองขวัญ เนื่องจากภายในเกมมีเนื้อหาที่รุนแรงภายในครอบครัว, การกลั่นแกล้งภายในโรงเรียนและการเพิกเฉยต่อการกลั่นแกล้ง
\end{itemize}

\subsection{\ifenglish Platform scope\else ขอบเขตด้านอุปกรณ์ใช้งาน\fi}
\begin{itemize}
    \item ใช้งานเฉพาะบนคอมพิวเตอร์ระบบปฏิบัติการ Window
\end{itemize}

\subsection{\ifenglish Gamplay scope\else ของเขตด้านเกมเพลย์\fi}
\begin{itemize}
    \item Side-scrolling game
    \item Single player
    \item Game offline
    \item Role play game
\end{itemize}

% \subsection{\ifenglish Story scope\else ขอบเขตด้านเนื้อหา\fi}
% \begin{itemize}
%     \item ผู้เล่นจะได้สวมบทบาทเป็นเพื่อนสนิทที่ต้องการหาความจริงเกี่ยวกับเพื่อนตัวเอง
%     % \item มีเนื้อหาความรุนแรงภายในครอบครัว,การกลั่นแกล้งภายในโรงเรียนและการเพิกเฉยต่อการกลั่นแกล้ง
% \end{itemize}

% \subsection{\ifenglish Hardware scope\else ขอบเขตด้านฮาร์ดแวร์\fi}
% \begin{itemize}
%     \item 
% \end{itemize}

% \subsection{\ifenglish Software scope\else ขอบเขตด้านซอฟต์แวร์\fi}
% \begin{itemize}
%     \item 
% \end{itemize}

\section{\ifenglish Expected outcomes\else ประโยชน์ที่ได้รับ\fi}
\begin{itemize}
    \item ความสยองขวัญ ความสนุก จากการที่ได้เล่น
    \item ข้อคิดดี ๆ จากเรื่องราวที่ผู้แต่งนำมาเสนอ
\end{itemize}

\section{\ifenglish Technology and tools\else เทคโนโลยีและเครื่องมือที่ใช้\fi}

% \subsection{\ifenglish Hardware technology\else เทคโนโลยีด้านฮาร์ดแวร์\fi}

\subsection{\ifenglish Software technology\else เทคโนโลยีด้านซอฟต์แวร์\fi}
\begin{itemize}
    \item Unity Engine ใช้ในการพัฒนาซอฟต์แวร์และการจำลองต่างๆ เช่น เกม เป็นต้น
    \item Visual Studio Community 2022 เครื่องที่ใช้ในการ Implement code
    \item C$\#$ language เป็นภาษาที่ใช้สำหรับการพัฒนาด้วย Unity
\end{itemize}

\section{\ifenglish Project plan\else แผนการดำเนินงาน\fi}

\begin{plan}{6}{2023}{3}{2024}
    \planitem{6}{2023}{6}{2023}{ออกแบบ main story }
    \planitem{6}{2023}{6}{2023}{จัดเตรียม tools ที่จะนำมาใช้ในการพัฒนา}
    \planitem{7}{2023}{12}{2023}{จัดเตรียม Assets ที่จะนำมาใช้ประกอบการพัฒนา}
    \planitem{7}{2023}{12}{2023}{จัดเตรียม animations ที่จะนำมาใช้ประกอบการพัฒนา}
    \planitem{7}{2023}{12}{2023}{จัดเตรียมแบบ form สำหรับเก็บข้อมูลจากผู้ใช้}
    \planitem{8}{2023}{9}{2023}{ออกแบบ story line }
    \planitem{8}{2023}{9}{2023}{ออกแบบและสร้าง Scene }
    \planitem{8}{2023}{10}{2023}{ออกแบบและสร้าง Script Enemy }
    \planitem{8}{2023}{12}{2023}{ออกแบบและสร้าง User Interface }
    \planitem{8}{2023}{9}{2023}{ออกแบบและสร้าง Script health, stamina}
    \planitem{8}{2023}{11}{2023}{ออกแบบและสร้าง Script sanity}
    \planitem{9}{2023}{9}{2023}{ออกสำรวจความคิดเห็นด้วย แบบสอบถาม}
    \planitem{11}{2023}{2}{2024}{ทำฉาก cut scene}
    \planitem{9}{2023}{1}{2024}{ออกแบบและสร้าง Script puzzle}
    \planitem{8}{2023}{2}{2024}{ออกแบบและสร้าง Script hide system}
    \planitem{1}{2024}{1}{2024}{ออกสำรวจความคิดเห็นด้วย Demo}
    \caption[ตารางการทำงานรายวิชา 261491]{ตารางแสดงการทำงานในรายวิชา 261491}
\end{plan}

\newpage
\section{\ifenglish Roles and responsibilities\else บทบาทและความรับผิดชอบ\fi}
% อธิบายว่าในการทำงาน นศ. มีการกำหนดบทบาทและแบ่งหน้าที่งานอย่างไรในการทำงาน จำเป็นต้องใช้ความรู้ใดในการทำงานบ้าง
\begin{itemize}
    \item[] 630612104 นายปริญญา ม่วงรอด รับหน้าที่ Developer
    \item[] 630612190 นางสาววิภาวี วรรธนัจฉริยา รับหน้าที่ Designer
\end{itemize}

\section{\ifenglish%
      Impacts of this project on society, health, safety, legal, and cultural issues
  \else%
      ผลกระทบด้านสังคม สุขภาพ ความปลอดภัย กฎหมาย และวัฒนธรรม
  \fi}
  เกมนี้มีการนำเสนอเนื้อหาของความรุนแรงภายในครอบครัว และโรงเรียน ซึ่งเป็นเรื่องที่พบได้มากในประเทศไทย 
  โดยภายในเกมจะนำเสนอมุมมองของเหยื่อที่ถูกกระทำ และความเจ็บปวดที่ได้รับ ผ่านเรื่องราวที่ได้แต่งขึ้นจากเค้าโครงเรื่องจริงจากผู้แต่ง
  ซึ่งหวังว่าจะทำให้ผู้เล่นตระหนักถึงเรื่องใจเขาใจเรา และตระหนักคิดก่อนลงมือทำสิ่งต่าง ๆ ได้มากยิ่งขึ้น

% แนวทางและโยชน์ในการประยุกต์ใช้งานโครงงานกับงานในด้านอื่นๆ รวมถึงผลกระทบในด้านสังคมและสิ่งแวดล้อมจากการใช้ความรู้ทางวิศวกรรมที่ได้
